\documentclass[oneside,a4paper,12pt]{article}
\usepackage[english,brazilian]{babel}
\usepackage[alf]{abntex2cite}
\usepackage[utf8]{inputenc}
\usepackage[T1]{fontenc}

\usepackage{lastpage}			  % Usado pela Ficha catalográfica
\usepackage{indentfirst}		  % Indenta o primeiro parágrafo de cada
\usepackage[top=20mm, bottom=20mm, left=20mm, right=20mm]{geometry}
\usepackage{framed}
\usepackage{booktabs}

\usepackage{float}
\usepackage{color}				  % Controle das cores
\usepackage{graphicx}			  % Inclusão de gráficos
\usepackage{microtype} 		      % para melhorias de justificação
\usepackage{booktabs}
\usepackage{multirow}
\usepackage[table]{xcolor}
\usepackage{subfig}
\usepackage{epstopdf}
\usepackage{hyperref}

\usepackage[mathcal]{eucal}
\usepackage{amsmath}               % great math stuff
\usepackage{amsfonts}              % for blackboard bold, etc
\usepackage{amsthm}                % better theorem environments
\usepackage{amssymb}
\usepackage{mathrsfs}
\DeclareMathAlphabet{\mathpzc}{OT1}{pzc}{m}{it}
\usepackage{undertilde}            % botar tilde embaixo da letra
\usepackage{mathptmx}          % fonte
\usepackage{graphicx}
\graphicspath{{./Figuras/}}    
\definecolor{shadecolor}{rgb}{0.8,0.8,0.8}


%FAZ EDICOES AQUI (somente no conteudo que esta entre entre as ultimas  chaves de cada linha!!!)
\newcommand{\universidade}{Universidade Federal de Pernambuco}
\newcommand{\centro}{Centro Acadêmico do Agreste}
\newcommand{\departamento}{Núcleo de Tecnologia}
\newcommand{\curso}{Engenharia Civil}
\newcommand{\professor}{Fernando R. L. Contreras}
\newcommand{\disciplina}{Algebra Linear}
%ATE AQUI !!!

\begin{document}
	\pagestyle{empty}
	
	\begin{center}
		%\includegraphics[width=\linewidth/6]{logoUFPE.jpg}%LOGOTIPO DA INSTITUICAO
	\vspace{0pt}
	
	\universidade
	\par
	\centro
	\par
	\departamento
	\par
	\curso
	\par
	\vspace{08pt}
		\text{Segunda Chamada}\\
		 \text{Segunda Prova - Algebra Linear}\\
		\text{Prof. Fernando R. L. Contreras}	
	\end{center}
	
	%\vspace{0.5pt}
	
	\begin{flushleft}
		Aluno(a):
	\end{flushleft}
	
\begin{itemize}
\item[1.] Seja $W\subset \mathbb{R}^{3}$ o subespaço gerado por $(1,0,1)$ e $(1,1,0)$. Considere $W^{\perp}$ em relação ao produto interno usual. Encontre uma base par $W^{\perp}$.
\end{itemize}
\begin{itemize}
\item[2.] Mostre que matriz $A=\begin{bmatrix}
1 & 2 \\
3 &2
\end{bmatrix}$ é semelhante à matriz $\begin{bmatrix}
4 & 0 \\
0 & -1
\end{bmatrix}$.
 \end{itemize}
 \begin{itemize}
\item [3.] Seja $Q(x,y)=x^{2}+12xy-4y^{2}$. Determine uma base $\beta$ tal que $[v]_{\beta}=\begin{bmatrix}
x_{1} \\
x_{2}
\end{bmatrix}$ e $Q(v)=ax_{1}^{2}+by_{1}^{2}$.
\end{itemize}
\begin{itemize}
\item[4.]Ache valores para $x$ e $y$ tais que $\left[\begin{array}{rr}
x&y\\
-1&0
\end{array}\right]$ seja uma matriz ortogonal.
\end{itemize}

	\flushbottom
	\flushright
     Êxitos...!!!
  \vspace{12pt}
  
     \begin{center}
     	%\includegraphics[width=\linewidth/6]{logoUFPE.jpg}%LOGOTIPO DA INSTITUICAO
     	\vspace{50pt}
     	
     	\universidade
     	\par
     	\centro
     	\par
     	\departamento
     	\par
     	\curso
     	\par
     	\vspace{08pt}
     	\text{Segunda Chamada}\\ \text{Primeira Prova - Algebra Linear}\\
     	\text{Prof. Fernando R. L. Contreras}	
     \end{center}
     
     \begin{flushleft}
     	Aluno(a):
     \end{flushleft}
 
 \begin{itemize}
 	\item[1.] Seja $T:\mathbb{R}^2\longrightarrow \mathbb{R}^2$ tal que $[T]=\begin{bmatrix}
 	-1& -2 \\
 	0& 1
 	\end{bmatrix}$. Ache os vetores $u$, $v$ tal que a) $T(u)=u$ e b) $T(v)=-v$.
 \end{itemize}
 \begin{itemize}
 	\item[2.] Mostre que $W=\left\lbrace (x,y,z)\in \mathbb{R}^{4}: x+y=0 \quad \text{e} \quad z-t=0\right\rbrace $ é subespaço.
 \end{itemize}
 \begin{itemize}
 	\item [3.] Considere dois vetores $(a,b)$ e $(c,d)$ no plano. Se $ad-bc=0$, mostre que eles são LD. $ad-bc\neq0$, mostre que eles são LI.
 \end{itemize}
 \begin{itemize}
 	\item[4.] Sejam $\beta=\{(1,0),(0,1)\}$, $\beta_{3}=\{(2,0), (0,2)\}$ bases ordenadas de $R^{2}$. Ache a matriz de mudança de base: $[I]_{\beta_{3}}^{\beta}$.
 	
 \end{itemize}
\flushbottom
\flushright
Êxitos...!!!
\end{document}

