\documentclass[oneside,a4paper,12pt]{article}
\usepackage[english,brazilian]{babel}
\usepackage[alf]{abntex2cite}
\usepackage[utf8]{inputenc}
\usepackage[T1]{fontenc}

\usepackage{lastpage}			  % Usado pela Ficha catalográfica
\usepackage{indentfirst}		  % Indenta o primeiro parágrafo de cada
\usepackage[top=20mm, bottom=20mm, left=20mm, right=20mm]{geometry}
\usepackage{framed}
\usepackage{booktabs}

\usepackage{float}
\usepackage{color}				  % Controle das cores
\usepackage{graphicx}			  % Inclusão de gráficos
\usepackage{microtype} 		      % para melhorias de justificação
\usepackage{booktabs}
\usepackage{multirow}
\usepackage[table]{xcolor}
\usepackage{subfig}
\usepackage{epstopdf}
\usepackage{hyperref}

\usepackage[mathcal]{eucal}
\usepackage{amsmath}               % great math stuff
\usepackage{amsfonts}              % for blackboard bold, etc
\usepackage{amsthm}                % better theorem environments
\usepackage{amssymb}
\usepackage{mathrsfs}
\DeclareMathAlphabet{\mathpzc}{OT1}{pzc}{m}{it}
\usepackage{undertilde}            % botar tilde embaixo da letra
\usepackage{mathptmx}          % fonte
\usepackage{graphicx}
\graphicspath{{./Figuras/}}    
\definecolor{shadecolor}{rgb}{0.8,0.8,0.8}


%FAZ EDICOES AQUI (somente no conteudo que esta entre entre as ultimas  chaves de cada linha!!!)
\newcommand{\universidade}{Universidade Federal de Pernambuco}
\newcommand{\centro}{Centro Acadêmico do Agreste}
\newcommand{\departamento}{Núcleo de Tecnologia}
\newcommand{\curso}{Engenharia Civil}
\newcommand{\professor}{Fernando R. L. Contreras}
\newcommand{\disciplina}{Algebra Linear}
%ATE AQUI !!!

\begin{document}
	\pagestyle{empty}
	
	\begin{center}
	%\includegraphics[width=\linewidth/6]{logoUFPE.jpg}%LOGOTIPO DA INSTITUICAO
	 	\vspace{0pt}
	 	
		\universidade
		\par
		\centro
		\par
		\departamento
		\par
		\curso
		\par
		\vspace{08pt}
		\text{Prova Final - Algebra Linear}\\
		\text{Prof. Fernando R. L. Contreras}	
	\end{center}
	
	%\vspace{0.5pt}
	
	\begin{flushleft}
		Aluno(a):
	\end{flushleft}
	
\begin{itemize}
\item[1.]Ache a transformação linear $T:R^{3}\longrightarrow R^{2}$ tal que $T(1,0,0)=(2,0)$, $T(0,1,0)=(1,1)$ e $T(0,0,1)=(0,-1)$.
\end{itemize}
\begin{itemize}
\item[2.] Seja $T:R^{2}\longrightarrow R^{2}$ uma transformação linear definida por $T(x,y)=(x+y,2x+y)$. Determine os autovetores de $T$. $T$ é diagonalizável? Justifique sua resposta e em caso afirmativo dê sua forma diagonal. 
 \end{itemize}
 \begin{itemize}
\item [3.] São sub-espaços vetoriais de $C(I)$ os seguintes subconjuntos: $U=\left\lbrace f\in C(I): f(t)=f(-t), \forall t\in \mathbb{R}\right\rbrace $ e $V= \left\lbrace f\in C(I): f(t)=-f(-t),\forall t\in \mathbb{R}\right\rbrace $. Mostra que $C(I)=U\bigoplus V$..
\end{itemize}
\begin{itemize}
\item[4.] Mostre que, se $A$ e $B$ são matrizes ortogonais, então $AB$ também é ortogonal.
\end{itemize}

	\flushbottom
	\flushright
     Êxitos...!!!
  \vspace{12pt}
  
     \begin{center}
     	%\includegraphics[width=\linewidth/6]{logoUFPE.jpg}%LOGOTIPO DA INSTITUICAO
     	\vspace{0pt}
     	
     	\universidade
     	\par
     	\centro
     	\par
     	\departamento
     	\par
     	\curso
     	\par
     	\vspace{08pt}
     	\text{Prova Final - Algebra Linear}\\
     	\text{Prof. Fernando R. L. Contreras}	
     \end{center}
     
     \begin{flushleft}
     	Aluno(a):
     \end{flushleft}
 
\begin{itemize}
	\item[1.]Ache a transformação linear $T:R^{3}\longrightarrow R^{2}$ tal que $T(1,0,0)=(2,0)$, $T(0,1,0)=(1,1)$ e $T(0,0,1)=(0,-1)$.
\end{itemize}
\begin{itemize}
	\item[2.] Seja $T:R^{2}\longrightarrow R^{2}$ uma transformação linear definida por $T(x,y)=(x+y,2x+y)$. Determine os autovetores de $T$. $T$ é diagonalizável? Justifique sua resposta e em caso afirmativo dê sua forma diagonal. 
\end{itemize}
\begin{itemize}
	\item [3.] Verifique se as aplicações abaixo são formas bilineares.\\
	$T:R^{2}\times R^{2}\longrightarrow R$ definida por:\\ (a) $T((x_{1},y_{1}),(x_{2},y_{2}))=x_{1}+y_{2}$ \\
	(b) $T((x_{1},y_{1}),(x_{2},y_{2}))=x_{1}x_{2}-2y_{1}y_{2}$.
\end{itemize}
\begin{itemize}
	\item[4.] Mostre que, se $A$ e $B$ são matrizes ortogonais, então $AB$ também é ortogonal.
\end{itemize}
\flushbottom
\flushright
Êxitos...!!!
\end{document}

