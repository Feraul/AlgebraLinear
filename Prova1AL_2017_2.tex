\documentclass[oneside,a4paper,12pt]{article}
\usepackage[english,brazilian]{babel}
\usepackage[alf]{abntex2cite}
\usepackage[utf8]{inputenc}
\usepackage[T1]{fontenc}

\usepackage{lastpage}			  % Usado pela Ficha catalográfica
\usepackage{indentfirst}		  % Indenta o primeiro parágrafo de cada
\usepackage[top=20mm, bottom=20mm, left=20mm, right=20mm]{geometry}
\usepackage{framed}
\usepackage{booktabs}

\usepackage{float}
\usepackage{color}				  % Controle das cores
\usepackage{graphicx}			  % Inclusão de gráficos
\usepackage{microtype} 		      % para melhorias de justificação
\usepackage{booktabs}
\usepackage{multirow}
\usepackage[table]{xcolor}
\usepackage{subfig}
\usepackage{epstopdf}
\usepackage{hyperref}

\usepackage[mathcal]{eucal}
\usepackage{amsmath}               % great math stuff
\usepackage{amsfonts}              % for blackboard bold, etc
\usepackage{amsthm}                % better theorem environments
\usepackage{amssymb}
\usepackage{mathrsfs}
\DeclareMathAlphabet{\mathpzc}{OT1}{pzc}{m}{it}
\usepackage{undertilde}            % botar tilde embaixo da letra
\usepackage{mathptmx}          % fonte
\usepackage{graphicx}
\graphicspath{{./Figuras/}}    
\definecolor{shadecolor}{rgb}{0.8,0.8,0.8}


%FAZ EDICOES AQUI (somente no conteudo que esta entre entre as ultimas  chaves de cada linha!!!)
\newcommand{\universidade}{Universidade Federal de Pernambuco}
\newcommand{\centro}{Centro Acadêmico do Agreste}
\newcommand{\departamento}{Núcleo de Tecnologia}
\newcommand{\curso}{Engenharia Civil}
\newcommand{\professor}{Fernando R. L. Contreras}
\newcommand{\disciplina}{Algebra Linear}
%ATE AQUI !!!

\begin{document}
	\pagestyle{empty}
	
	\begin{center}
	%\includegraphics[width=\linewidth/6]{logoUFPE.jpg}%LOGOTIPO DA INSTITUICAO
	 	\vspace{0pt}
	 	
		\universidade
		\par
		\centro
		\par
		\departamento
		\par
		\curso
		\par
		\vspace{08pt}
		\text{Prova 1 - Algebra Linear}\\
		\text{Prof. Fernando R. L. Contreras}	
	\end{center}
	
	%\vspace{0.5pt}
	
	\begin{flushleft}
		Aluno(a):
	\end{flushleft}
	
\begin{itemize}
\item[1.]Considere a transformação linear $T:\mathbb{R}^{3}\longrightarrow \mathbb{R}^{3}$ dada por $T(x,y,z)=(z,x-y,-z)$. (a) Determine uma base do núcleo de $T$. (b) Dê a dimensão da imagem de $T$. (c) T é sobrejetora? Justifique.
\end{itemize}
\begin{itemize}
\item[2.] Mostre que os polinômios $1-t^{3}$, $(1-t)^{2}$, $1-t$ e 1 geram o espaço dos polinômios de grau $\leqslant$ 3. 
 \end{itemize}
 \begin{itemize}
\item [3.]  Sejam $\beta=\left\lbrace (1,0),(0,1) \right\rbrace $, $\beta_{1}=\left\lbrace (-1,1), (1,1)\right\rbrace $ bases ordenadas de $\mathbb{R}^{2}$. Ache a matriz de mudança de base $[I]_{\beta_{1}}^{\beta}$. E quais são as coordenadas do vetor $v=(3,-2)$ em relação à base $\beta_{1}$.
\end{itemize}
\begin{itemize}
\item[4.] Sejam $\alpha=\left\lbrace (1,-1),(0,2)\right\rbrace $ e $\beta=\left\lbrace (1,0,-1),(0,1,2),(1,2,0)\right\rbrace $ bases de $\mathbb{R}^{2}$ e $\mathbb{R}^{3}$ respectivamente e
$[T]_{\beta}^{\alpha}=\left[\begin{array}{rr}
1&0\\
1&1\\
0&-1
\end{array}\right]$. Ache $T$.

\end{itemize}

	
  \vspace{60pt}
  
     \begin{center}
     	%\includegraphics[width=\linewidth/6]{logoUFPE.jpg}%LOGOTIPO DA INSTITUICAO
     	\vspace{0pt}
     	
     	\universidade
     	\par
     	\centro
     	\par
     	\departamento
     	\par
     	\curso
     	\par
     	\vspace{08pt}
     	\text{Prova 1 - Algebra Linear}\\
     	\text{Prof. Fernando R. L. Contreras}	
     \end{center}
     
     \begin{flushleft}
     	Aluno(a):
     \end{flushleft}
 
\begin{itemize}
	\item[1.]Considere a transformação linear $T:\mathbb{R}^{3}\longrightarrow \mathbb{R}^{3}$ dada por $T(x,y,z)=(z,x-y,-z)$. (a) Determine uma base do núcleo de $T$. (b) Dê a dimensão da imagem de $T$. (c) T é sobrejetora? Justifique.
\end{itemize}
\begin{itemize}
	\item[2.] Mostre que os polinômios $1-t^{3}$, $(1-t)^{2}$, $1-t$ e 1 geram o espaço dos polinômios de grau $\leqslant$ 3. 
\end{itemize}
\begin{itemize}
	\item [3.] Sejam $\beta=\left\lbrace (1,0),(0,1) \right\rbrace $, $\beta_{1}=\left\lbrace (-1,1), (1,1)\right\rbrace $ bases ordenadas de $\mathbb{R}^{2}$. Ache a matriz de mudança de base $[I]_{\beta_{1}}^{\beta}$. E quais são as coordenadas do vetor $v=(3,-2)$ em relação à base $\beta_{1}$.
\end{itemize}
\begin{itemize}
	\item[4.] Sejam $\alpha=\left\lbrace (1,-1),(0,2)\right\rbrace $ e $\beta=\left\lbrace (1,0,-1),(0,1,2),(1,2,0)\right\rbrace $ bases de $\mathbb{R}^{2}$ e $\mathbb{R}^{3}$ respectivamente e
	$[T]_{\beta}^{\alpha}=\left[\begin{array}{rr}
	1&0\\
	1&1\\
	0&-1
	\end{array}\right]$. Ache $T$.
\end{itemize}

\end{document}

