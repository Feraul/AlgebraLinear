\documentclass[oneside,a4paper,12pt]{article}
\usepackage[english,brazilian]{babel}
\usepackage[alf]{abntex2cite}
\usepackage[utf8]{inputenc}
\usepackage[T1]{fontenc}

\usepackage{lastpage}			  % Usado pela Ficha catalográfica
\usepackage{indentfirst}		  % Indenta o primeiro parágrafo de cada
\usepackage[top=20mm, bottom=20mm, left=20mm, right=20mm]{geometry}
\usepackage{framed}
\usepackage{booktabs}

\usepackage{float}
\usepackage{color}				  % Controle das cores
\usepackage{graphicx}			  % Inclusão de gráficos
\usepackage{microtype} 		      % para melhorias de justificação
\usepackage{booktabs}
\usepackage{multirow}
\usepackage[table]{xcolor}
\usepackage{subfig}
\usepackage{epstopdf}
\usepackage{hyperref}

\usepackage[mathcal]{eucal}
\usepackage{amsmath}               % great math stuff
\usepackage{amsfonts}              % for blackboard bold, etc
\usepackage{amsthm}                % better theorem environments
\usepackage{amssymb}
\usepackage{mathrsfs}
\DeclareMathAlphabet{\mathpzc}{OT1}{pzc}{m}{it}
\usepackage{undertilde}            % botar tilde embaixo da letra
\usepackage{mathptmx}          % fonte
\usepackage{graphicx}
\graphicspath{{./Figuras/}}    
\definecolor{shadecolor}{rgb}{0.8,0.8,0.8}

%FAZ EDICOES AQUI (somente no conteudo que esta entre entre as ultimas  chaves de cada linha!!!)
\newcommand{\universidade}{Universidade Federal de Pernambuco}
\newcommand{\centro}{Centro Acadêmico do Agreste}
\newcommand{\departamento}{Núcleo de Tecnologia}
\newcommand{\curso}{Engenharia Civil}
\newcommand{\professor}{Fernando R. L. Contreras}
\newcommand{\disciplina}{Algebra Linear}
%ATE AQUI !!!

\begin{document}
	\pagestyle{empty}
	
	\begin{center}
	%\includegraphics[width=\linewidth/6]{logoUFPE.jpg}%LOGOTIPO DA INSTITUICAO
	 	\vspace{0pt}
	 	
		\universidade
		\par
		\centro
		\par
		\departamento
		\par
		\curso
		\par
		\vspace{08pt}
		\text{Prova 1 - Algebra Linear}\\
		\text{Prof. Fernando R. L. Contreras}	
	\end{center}
	
	%\vspace{0.5pt}
	
	\begin{flushleft}
		\textbf{Aluno(a)}:
	\end{flushleft}
	
\begin{itemize}
\item[1.]  Seja $T:P_{3}\longrightarrow P_{3} $ tal que $T(f)=f''$ para todo $f\in P_{3}$. Mostre que $T$ é transformação linear e determine uma base para $Ker(T)$.
\end{itemize}
\begin{itemize}
\item[2.] Sejam $\alpha=\left\lbrace (1,-1),(0,2) \right\rbrace $ e $\beta=\left\lbrace (1,0,-1),(0,1,2), (1,2,0)\right\rbrace $ bases de $\mathbb{R}^{2} $ e $\mathbb{R}^{3}$ respectivamente e\\ $[T]^{\alpha}_{\beta}=\begin{bmatrix}
	1       & 0 \\
	1       & 1 \\
	0       & -1
\end{bmatrix}$. (a) Ache $T$ e (b) Se $S(x,y)=(2y,x-y,x)$, ache $[S]^{\alpha}_{\beta}$
 \end{itemize}
 \begin{itemize}
\item [3.] Dados os subespaços de $\mathbb{R}^{4}$, $S_{1}=\left\lbrace (x_{1},x_{2},x_{3},x_{4}) / x_{1}+x_{2}-x_{3}+x_{4}=0 \right\rbrace $ e\\ $S_{2}=\left\lbrace (x_{1},x_{2},x_{3},x_{4}) / x_{1}-x_{2}-x_{3}-x_{4}=0 \right\rbrace $. Obter a dimensão de $S_{1}+S_{2}$.
\end{itemize}
\begin{itemize}
\item[4.]. No conjunto $V$ definamos a "\textit{adição}" como fazemos habitualmente no $ \mathbb{R}^{2}$ e multiplicação por escalares assim: $\alpha (x,y)=(ax,0)$. É então $V$ um espaço vetorial sobre $ \mathbb{R}$? Por que ?

\end{itemize}
\begin{itemize}
	\item[Opcional]. Defina a Soma Direta de dois subespaços vetoriais do espaço vetorial $V$.
	
\end{itemize}
	\flushbottom
	\flushright
    
  \vspace{12pt}
  
     \begin{center}
     	%\includegraphics[width=\linewidth/6]{logoUFPE.jpg}%LOGOTIPO DA INSTITUICAO
     	\vspace{0pt}
     	
     	\universidade
     	\par
     	\centro
     	\par
     	\departamento
     	\par
     	\curso
     	\par
     	\vspace{08pt}
     	\text{Prova 1 - Algebra Linear}\\
     	\text{Prof. Fernando R. L. Contreras}	
     \end{center}
     
     \begin{flushleft}
     	\textbf{Aluno(a)}:
     \end{flushleft}
 
 \begin{itemize}
 	\item[1.]  Seja $T:P_{3}\longrightarrow P_{3} $ tal que $T(f)=f''$ para todo $f\in P_{3}$. Mostre que $T$ é transformação linear e determine uma base para $Ker(T)$.
 \end{itemize}
 \begin{itemize}
 	\item[2.] Sejam $\alpha=\left\lbrace (1,-1),(0,2) \right\rbrace $ e $\beta=\left\lbrace (1,0,-1),(0,1,2), (1,2,0)\right\rbrace $ bases de $\mathbb{R}^{2} $ e $\mathbb{R}^{3}$ respectivamente e\\ $[T]^{\alpha_{\beta}}=\begin{bmatrix}
 	1       & 0 \\
 	1       & 1 \\
 	0       & -1
 	\end{bmatrix}$. (a) Ache $T$ e (b) Se $S(x,y)=(2y,x-y,x)$, ache $[S]^{\alpha}_{\beta}$
 \end{itemize}
 \begin{itemize}
 	\item [3.] Dados os subespaços de $\mathbb{R}^{4}$, $S_{1}=\left\lbrace (x_{1},x_{2},x_{3},x_{4}) / x_{1}+x_{2}-x_{3}+x_{4}=0 \right\rbrace $ e\\ $S_{2}=\left\lbrace (x_{1},x_{2},x_{3},x_{4}) / x_{1}-x_{2}-x_{3}-x_{4}=0 \right\rbrace $. Obter a dimensão de $S_{1}+S_{2}$.
 \end{itemize}
 \begin{itemize}
 	\item[4.]. No conjunto $V$ definamos a "\textit{adição}" como fazemos habitualmente no $ \mathbb{R}^{2}$ e multiplicação por escalares assim: $\alpha (x,y)=(ax,0)$. É então $V$ um espaço vetorial sobre $ \mathbb{R}$? Por que ?
\end{itemize}
\begin{itemize}
	\item[Opcional]. Defina a Soma Direta de dois subespaços vetoriais do espaço vetorial $V$.
	
\end{itemize}	
\flushbottom
\flushright

\end{document}

