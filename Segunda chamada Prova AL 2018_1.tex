\documentclass[oneside,a4paper,12pt]{article}
\usepackage[english,brazilian]{babel}
\usepackage[alf]{abntex2cite}
\usepackage[utf8]{inputenc}
\usepackage[T1]{fontenc}

\usepackage{lastpage}			  % Usado pela Ficha catalográfica
\usepackage{indentfirst}		  % Indenta o primeiro parágrafo de cada
\usepackage[top=20mm, bottom=20mm, left=20mm, right=20mm]{geometry}
\usepackage{framed}
\usepackage{booktabs}

\usepackage{float}
\usepackage{color}				  % Controle das cores
\usepackage{graphicx}			  % Inclusão de gráficos
\usepackage{microtype} 		      % para melhorias de justificação
\usepackage{booktabs}
\usepackage{multirow}
\usepackage[table]{xcolor}
\usepackage{subfig}
\usepackage{epstopdf}
\usepackage{hyperref}

\usepackage[mathcal]{eucal}
\usepackage{amsmath}               % great math stuff
\usepackage{amsfonts}              % for blackboard bold, etc
\usepackage{amsthm}                % better theorem environments
\usepackage{amssymb}
\usepackage{mathrsfs}
\DeclareMathAlphabet{\mathpzc}{OT1}{pzc}{m}{it}
\usepackage{undertilde}            % botar tilde embaixo da letra
\usepackage{mathptmx}          % fonte
\usepackage{graphicx}
\graphicspath{{./Figuras/}}    
\definecolor{shadecolor}{rgb}{0.8,0.8,0.8}


%FAZ EDICOES AQUI (somente no conteudo que esta entre entre as ultimas  chaves de cada linha!!!)
\newcommand{\universidade}{Universidade Federal de Pernambuco}
\newcommand{\centro}{Centro Acadêmico do Agreste}
\newcommand{\departamento}{Núcleo de Tecnologia}
\newcommand{\curso}{Engenharia Civil}
\newcommand{\professor}{Fernando R. L. Contreras}
\newcommand{\disciplina}{Algebra Linear}
%ATE AQUI !!!

\begin{document}
	\pagestyle{empty}
	
	\begin{center}
	%\includegraphics[width=\linewidth/6]{logoUFPE.jpg}%LOGOTIPO DA INSTITUICAO
	 	\vspace{0pt}
	 	
		\universidade
		\par
		\centro
		\par
		\departamento
		\par
		\curso
		\par
		\vspace{08pt}
		\text{Segunda Chamada}\\
		 \text{Primeira Prova - Algebra Linear}\\
		\text{Prof. Fernando R. L. Contreras}	
	\end{center}
	
	%\vspace{0.5pt}
	
	\begin{flushleft}
		Aluno(a):
	\end{flushleft}
	
\begin{itemize}
\item[1.] Seja $T:P_{3}\longrightarrow P_{3} $ tal que $T(f)=f''$ para todo $f\in P_{3}$. Mostre que $T$ é transformação linear e determine uma base para $Ker(T)$.
\end{itemize}
\begin{itemize}
\item[2.] Ache a transformação linear $T:R^{3}\longrightarrow R^{2}$ tal que $T(1,0,0)=(2,0)$, $T(0,1,0)=(1,1)$ e $T(0,0,1)=(0,-1)$.
 \end{itemize}
 \begin{itemize}
\item [3.] Sejam $\beta=\{(1,0),(0,1)\}$, $\beta_{1}=\{(-1,1), (1,1)\}$ bases ordenadas de $R^{2}$. Ache a matriz de mudança de base: $[I]_{\beta}^{\beta_{1}}$.
\end{itemize}
\begin{itemize}
\item[4.]São sub-espaços vetoriais de $C(I)$ os seguintes subconjuntos: $U=\left\lbrace f\in C(I): f(t)=f(-t), \forall t\in \mathbb{R}\right\rbrace $ e $V= \left\lbrace f\in C(I): f(t)=-f(-t),\forall t\in \mathbb{R}\right\rbrace $. Mostra que $C(I)=U\bigoplus V$.
\end{itemize}
\begin{itemize}
	\item[Opcional]. Enuncie o Teorema de Núcleo e Imagem.
	
\end{itemize}

  \vspace{12pt}
  
     \begin{center}
     	%\includegraphics[width=\linewidth/6]{logoUFPE.jpg}%LOGOTIPO DA INSTITUICAO
     	\vspace{0pt}
     	
     	\universidade
     	\par
     	\centro
     	\par
     	\departamento
     	\par
     	\curso
     	\par
     	\vspace{08pt}
     	\text{Segunda Chamada}\\ \text{Terceira Prova - Algebra Linear}\\
     	\text{Prof. Fernando R. L. Contreras}	
     \end{center}
     
     \begin{flushleft}
     	Aluno(a):
     \end{flushleft}
 
 \begin{itemize}
 	\item[1.] Identificar a seguinte quádrica $3x^{2}+5y^{2}+3z^{2}-2xy+2xz-2yz-4x+6y-2z+2=0$ utilizando os conceitos álgebra linear.
 \end{itemize}
 \begin{itemize}
 	\item[2.] Determinar a equação reduzida e esboce o gráfico da cônica representada pela equação: $7x^{2}+13y^{2}-6\sqrt{3}xy-16=0$.
 \end{itemize}
 \begin{itemize}
 	\item [3.] Dê a solução geral do seguinte sistema de equações diferenciais
 	$\begin{cases}
 	\frac{dx}{dt}=y+z,\\
 	\frac{dy}{dt}=x+z,\\
 	\frac{dy}{dt}=x+y
 	\end{cases}$.
 \end{itemize}
 \begin{itemize}
 	\item[4.]  Se $A$ é uma forma bilinear simétrica e $Q$ a forma quadrática associada a ela mostre que $A(v,w)=\frac{1}{2}(Q(v+w)-Q(v)- Q(w))$.
 	
 \end{itemize}
\begin{itemize}
	\item[Opcional.] Seja  $g:V\times V \longrightarrow \mathbb{R}$ uma forma bilinear. Demostrar que $g_{v}: V \longrightarrow \mathbb{R} $, definida por $g_{v}(u)=g(u,v)$ é uma forma linear.
	
\end{itemize}
\end{document}

