\documentclass[oneside,a4paper,12pt]{article}
\usepackage[english,brazilian]{babel}
\usepackage[alf]{abntex2cite}
\usepackage[utf8]{inputenc}
\usepackage[T1]{fontenc}

\usepackage{lastpage}			  % Usado pela Ficha catalográfica
\usepackage{indentfirst}		  % Indenta o primeiro parágrafo de cada
\usepackage[top=20mm, bottom=20mm, left=20mm, right=20mm]{geometry}
\usepackage{framed}
\usepackage{booktabs}

\usepackage{float}
\usepackage{color}				  % Controle das cores
\usepackage{graphicx}			  % Inclusão de gráficos
\usepackage{microtype} 		      % para melhorias de justificação
\usepackage{booktabs}
\usepackage{multirow}
\usepackage[table]{xcolor}
\usepackage{subfig}
\usepackage{epstopdf}
\usepackage{hyperref}

\usepackage[mathcal]{eucal}
\usepackage{amsmath}               % great math stuff
\usepackage{amsfonts}              % for blackboard bold, etc
\usepackage{amsthm}                % better theorem environments
\usepackage{amssymb}
\usepackage{mathrsfs}
\DeclareMathAlphabet{\mathpzc}{OT1}{pzc}{m}{it}
\usepackage{undertilde}            % botar tilde embaixo da letra
\usepackage{mathptmx}          % fonte
\usepackage{graphicx}
\graphicspath{{./Figuras/}}    
\definecolor{shadecolor}{rgb}{0.8,0.8,0.8}

%FAZ EDICOES AQUI (somente no conteudo que esta entre entre as ultimas  chaves de cada linha!!!)
\newcommand{\universidade}{Universidade Federal de Pernambuco}
\newcommand{\centro}{Centro Acadêmico do Agreste}
\newcommand{\departamento}{Núcleo de Tecnologia}
\newcommand{\curso}{Engenharia Civil}
\newcommand{\professor}{Fernando R. L. Contreras}
\newcommand{\disciplina}{Algebra Linear}
%ATE AQUI !!!

\begin{document}
	\pagestyle{empty}
	
	\begin{center}
	%\includegraphics[width=\linewidth/6]{logoUFPE.jpg}%LOGOTIPO DA INSTITUICAO
	 	\vspace{-50pt}
	 	
		\universidade
		\par
		\centro
		\par
		\departamento
		\par
		\curso
		\par
		\vspace{0pt}
		\text{Prova 2 - Algebra Linear}\\
		\text{Prof. Fernando R. L. Contreras}	
	\end{center}

	\begin{flushleft}
		\textbf{Aluno(a)}:
	\end{flushleft}
	
\begin{itemize}
\item[1.] Seja $V$ um espaço euclidiano. Demostrar que a função $N:V\rightarrow \mathbb{R}$ definida por $N(u)=<u,u>$. Verifica que:
\begin{itemize}
	\item $N(u+v)-N(u)-N(v)=2<u,v>$.
	\item $\frac{1}{4}N(u+v)-\frac{1}{4}N(u-v)=<u,v>$.
\end{itemize}
\end{itemize}
\begin{itemize}
\item[2.]  Mostre que $A=\begin{bmatrix}
1    & 2 \\
3    & 2
\end{bmatrix}$ é semelhante à matriz $B=\begin{bmatrix}
4    & 0 \\
0    & -1
\end{bmatrix}$.
 \end{itemize}
 \begin{itemize}
\item [3.] Consideremos o $\mathbb{R}^{4}$ munido do produto interno usual e seja $W=\{(x,y,z,t)\in \mathbb{R}^{4}|\quad x+y=0 \quad\text{e}\quad 2x-y+z=0\}$. Determine uma base ortonormal de $W$ e uma base ortonormal de $W^{\bot}$.
\end{itemize}
\begin{itemize}
\item[4.]. Dada a matriz $A=\begin{bmatrix}
a    & 0 & 0 \\
0    & b & c \\
0    & c & b
\end{bmatrix}$. Verifique que os autovalores são: $a$, $b+c$ e $b-c$. E ache uma base de autovetores.

\end{itemize}
\begin{itemize}
	\item[Opcional]. Forneça uma base distinto da base canônica em $\mathbb{R}^{2}$, e obtenha uma base ortogonal utilizando o processo de ortogonalização de Gram-Schmdit.
	
\end{itemize}
%\newpage
	\flushbottom
	\flushright
    
  \vspace{5pt}
  
     \begin{center}
     	%\includegraphics[width=\linewidth/6]{logoUFPE.jpg}%LOGOTIPO DA INSTITUICAO
     	\vspace{0pt}
     	
     	\universidade
     	\par
     	\centro
     	\par
     	\departamento
     	\par
     	\curso
     	\par
     	\vspace{0pt}
     	\text{Prova 2 - Algebra Linear}\\
     	\text{Prof. Fernando R. L. Contreras}	
     \end{center}
     
     \begin{flushleft}
     	\textbf{Aluno(a)}:
     \end{flushleft}
 
 \begin{itemize}
 	\item[1.] Seja $V$ um espaço euclidiano. Demostrar que a função $N:V\rightarrow \mathbb{R}$ definida por $N(u)=<u,u>$. Verifica que:
 	\begin{itemize}
 		\item $N(u+v)-N(u)-N(v)=2<u,v>$.
 		\item $\frac{1}{4}N(u+v)-\frac{1}{4}N(u-v)=<u,v>$.
 	\end{itemize}
 \end{itemize}
 \begin{itemize}
 	\item[2.]  Mostre que $A=\begin{bmatrix}
 	1    & 2 \\
 	3    & 2
 	\end{bmatrix}$ é semelhante à matriz $B=\begin{bmatrix}
 	4    & 0 \\
 	0    & -1
 	\end{bmatrix}$.
 \end{itemize}
 \begin{itemize}
 	\item [3.] Consideremos o $\mathbb{R}^{4}$ munido do produto interno usual e seja $W=\{(x,y,z,t)\in \mathbb{R}^{4}|\quad x+y=0 \quad\text{e}\quad 2x-y+z=0\}$. Determine uma base ortonormal de $W$ e uma base ortonormal de $W^{\bot}$.
 \end{itemize}
 \begin{itemize}
 	\item[4.]. Dada a matriz $A=\begin{bmatrix}
 	a    & 0 & 0 \\
 	0    & b & c \\
 	0    & c & b
 	\end{bmatrix}$. Verifique que os autovalores são: $a$, $b+c$ e $b-c$. E ache uma base de autovetores.
 	
 \end{itemize}
\begin{itemize}
	\item[Opcional]. Forneça uma base distinto da base canônica em $\mathbb{R}^{2}$, e obtenha uma base ortogonal utilizando o processo de ortogonalização de Gram-Schmdit.
	
\end{itemize}	
\flushbottom
\flushright

\end{document}

