\documentclass[oneside,a4paper,12pt]{article}
\usepackage[english,brazilian]{babel}
\usepackage[alf]{abntex2cite}
\usepackage[utf8]{inputenc}
\usepackage[T1]{fontenc}

\usepackage{lastpage}			  % Usado pela Ficha catalográfica
\usepackage{indentfirst}		  % Indenta o primeiro parágrafo de cada
\usepackage[top=20mm, bottom=20mm, left=20mm, right=20mm]{geometry}
\usepackage{framed}
\usepackage{booktabs}

\usepackage{float}
\usepackage{color}				  % Controle das cores
\usepackage{graphicx}			  % Inclusão de gráficos
\usepackage{microtype} 		      % para melhorias de justificação
\usepackage{booktabs}
\usepackage{multirow}
\usepackage[table]{xcolor}
\usepackage{subfig}
\usepackage{epstopdf}
\usepackage{hyperref}

\usepackage[mathcal]{eucal}
\usepackage{amsmath}               % great math stuff
\usepackage{amsfonts}              % for blackboard bold, etc
\usepackage{amsthm}                % better theorem environments
\usepackage{amssymb}
\usepackage{mathrsfs}
\DeclareMathAlphabet{\mathpzc}{OT1}{pzc}{m}{it}
\usepackage{undertilde}            % botar tilde embaixo da letra
\usepackage{mathptmx}          % fonte
\usepackage{graphicx}
\graphicspath{{./Figuras/}}    
\definecolor{shadecolor}{rgb}{0.8,0.8,0.8}

%FAZ EDICOES AQUI (somente no conteudo que esta entre entre as ultimas  chaves de cada linha!!!)
\newcommand{\universidade}{Universidade Federal de Pernambuco}
\newcommand{\centro}{Centro Acadêmico do Agreste}
\newcommand{\departamento}{Núcleo de Tecnologia}
\newcommand{\curso}{Engenharia Civil}
\newcommand{\professor}{Fernando R. L. Contreras}
\newcommand{\disciplina}{Algebra Linear}
%ATE AQUI !!!

\begin{document}
	\pagestyle{empty}
	
	\begin{center}
	%\includegraphics[width=\linewidth/6]{logoUFPE.jpg}%LOGOTIPO DA INSTITUICAO
	 	\vspace{0pt}
	 	
		\universidade
		\par
		\centro
		\par
		\departamento
		\par
		\curso
		\par
		\vspace{08pt}
		\text{Segunda Prova - Algebra Linear}\\
		\text{Prof. Fernando R. L. Contreras}	
	\end{center}
	
	%\vspace{0.5pt}
	
	\begin{flushleft}
		Aluno(a):
	\end{flushleft}
	
\begin{itemize}
\item[1.] Um corpo é deslocado em linha reta do ponto $(-1,3)$ até o ponto $(5,2)$ por uma força constante $F=(3,2)$. Qual é o trabalho realizado?
\end{itemize}
\begin{itemize}
\item[2.] Suponha que $v\in V$ seja autovetor de $T:V\longrightarrow V$ e $S:V\longrightarrow V$, ao mesmo tempo com autovalores $\lambda_{1}$ e $\lambda_{2}$ respectivamente. Ache autovetores e autovalores de $S+T$ e $S\circ T$.
 \end{itemize}
 \begin{itemize}
\item [3.] Seja $T(x,y,z)=(2x+y,x+y+z,y-3z)$ de $\mathbb{R}^{3}$ em $\mathbb{R}^{3}$ com produto interno canônico. Mostre que $T$ é um operador auto-adjunto mas não ortogonal. Se $v=(2,-1,5)$ e $w=(3,0,1)$, verifique que $\langle Tv,w \rangle=\langle v,Tw \rangle$.
\end{itemize}
\begin{itemize}
\item[4.] Se $A$ é uma forma bilinear simétrica e $Q$ a forma quadrática associada a ela, mostre que $A(v,w)=\frac{1}{4}Q(v+w)-\frac{1}{4}Q(v-w)$.
\end{itemize}

	\flushbottom
	\flushright
     Êxitos...!!!
  \vspace{12pt}
  
     \begin{center}
     	%\includegraphics[width=\linewidth/6]{logoUFPE.jpg}%LOGOTIPO DA INSTITUICAO
     	\vspace{0pt}
     	
     	\universidade
     	\par
     	\centro
     	\par
     	\departamento
     	\par
     	\curso
     	\par
     	\vspace{08pt}
     	\text{Segunda Prova - Algebra Linear}\\
     	\text{Prof. Fernando R. L. Contreras}	
     \end{center}
     
     \begin{flushleft}
     	Aluno(a):
     \end{flushleft}
 
 \begin{itemize}
 	\item[1.]  Um corpo é deslocado em linha reta do ponto $(-1,3)$ até o ponto $(5,2)$ por uma força constante $F=(3,2)$. Qual é o trabalho realizado?
 \end{itemize}
 \begin{itemize}
 	\item[2.] Suponha que $v\in V$ seja autovetor de $T:V\longrightarrow V$ e $S:V\longrightarrow V$, ao mesmo tempo com autovalores $\lambda_{1}$ e $\lambda_{2}$ respectivamente. Ache autovetores e autovalores de $S+T$ e $S\circ T$.
 \end{itemize}
 \begin{itemize}
 	\item [3.] Seja $T(x,y,z)=(2x+y,x+y+z,y-3z)$ de $\mathbb{R}^{3}$ em $\mathbb{R}^{3}$ com produto interno canônico. Mostre que $T$ é um operador auto-adjunto mas não ortogonal. Se $v=(2,-1,5)$ e $w=(3,0,1)$, verifique que $\langle Tv,w \rangle=\langle v,Tw \rangle$.
 \end{itemize}
 \begin{itemize}
 	\item[4.] Se $A$ é uma forma bilinear simétrica e $Q$ a forma quadrática associada a ela, mostre que $A(v,w)=\frac{1}{4}Q(v+w)-\frac{1}{4}Q(v-w)$.
 	
 \end{itemize}
 
\flushbottom
\flushright
Êxitos...!!!
\end{document}

