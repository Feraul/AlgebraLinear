\documentclass[oneside,a4paper,12pt]{article}
\usepackage[english,brazilian]{babel}
\usepackage[alf]{abntex2cite}
\usepackage[utf8]{inputenc}
\usepackage[T1]{fontenc}

\usepackage{lastpage}			  % Usado pela Ficha catalográfica
\usepackage{indentfirst}		  % Indenta o primeiro parágrafo de cada
\usepackage[top=20mm, bottom=20mm, left=20mm, right=20mm]{geometry}
\usepackage{framed}
\usepackage{booktabs}

\usepackage{float}
\usepackage{color}				  % Controle das cores
\usepackage{graphicx}			  % Inclusão de gráficos
\usepackage{microtype} 		      % para melhorias de justificação
\usepackage{booktabs}
\usepackage{multirow}
\usepackage[table]{xcolor}
\usepackage{subfig}
\usepackage{epstopdf}
\usepackage{hyperref}

\usepackage[mathcal]{eucal}
\usepackage{amsmath}               % great math stuff
\usepackage{amsfonts}              % for blackboard bold, etc
\usepackage{amsthm}                % better theorem environments
\usepackage{amssymb}
\usepackage{mathrsfs}
\DeclareMathAlphabet{\mathpzc}{OT1}{pzc}{m}{it}
\usepackage{undertilde}            % botar tilde embaixo da letra
\usepackage{mathptmx}          % fonte
\usepackage{graphicx}
\graphicspath{{./Figuras/}}    
\definecolor{shadecolor}{rgb}{0.8,0.8,0.8}

%FAZ EDICOES AQUI (somente no conteudo que esta entre entre as ultimas  chaves de cada linha!!!)
\newcommand{\universidade}{Universidade Federal de Pernambuco}
\newcommand{\centro}{Centro Acadêmico do Agreste}
\newcommand{\departamento}{Núcleo de Tecnologia}
\newcommand{\curso}{Engenharia Civil}
\newcommand{\professor}{Fernando R. L. Contreras}
\newcommand{\disciplina}{Algebra Linear}
%ATE AQUI !!!

\begin{document}
	\pagestyle{empty}
	
	\begin{center}
	%\includegraphics[width=\linewidth/6]{logoUFPE.jpg}%LOGOTIPO DA INSTITUICAO
	 	\vspace{0pt}
	 	
		\universidade
		\par
		\centro
		\par
		\departamento
		\par
		\curso
		\par
		\vspace{08pt}
		\text{Prova 3 - Algebra Linear}\\
		\text{Prof. Fernando R. L. Contreras}	
	\end{center}
	
	%\vspace{0.5pt}
	
	\begin{flushleft}
		\textbf{Aluno(a)}:
	\end{flushleft}
	
\begin{itemize}
\item[1.] Determinar a equação reduzida e esboce o gráfico da cônica representada pela equação: \\$11x^{2}-24xy+4y^{2}+20x-40y-20=0$.
\end{itemize}
\begin{itemize}
\item[2.] Identificar a seguinte quádrica $7x^{2}+6y^{2}+5z^{2}-4xy-4yz-18=0$ utilizando os conceitos álgebra linear.
 \end{itemize}
 \begin{itemize}
\item [3.] Dê a solução geral do seguinte sistema de equações diferenciais
$\begin{cases}
\frac{dx}{dt}=3x-4y,\\
\frac{dy}{dt}=x-y
\end{cases}$.
\end{itemize}
\begin{itemize}
\item[4.]. Seja $V$ o espaço vetorial de dimensão finita e sejam as funções $F:V\longrightarrow \mathbb{R}$ e $G: V \times V \longrightarrow \mathbb{R}$ tal que: $G(u,v)=F(u+v)-F(u)-F(v)$. Supondo que $G$ é forma bilinear e que $F(\alpha u)=\alpha^{2}F(u)$, para todo $\alpha\in \mathbb{R}$ e $u\in V$, demostrar que $F$ é uma forma quadrática e determinar a forma bilinear da qual provém.
\end{itemize}
\begin{itemize}
	\item[Opcional]. Uma forma quadrática $T$ é chamada positiva definida, se para todo $v\neq 0$, $T(v)>0$. Como devem ser os autovalores da matriz de uma forma quadrática positiva definida? \textbf{Justifique.}
	
\end{itemize}
	\flushbottom
	\flushright
    
  \vspace{12pt}
  
     \begin{center}
     	%\includegraphics[width=\linewidth/6]{logoUFPE.jpg}%LOGOTIPO DA INSTITUICAO
     	\vspace{0pt}
     	
     	\universidade
     	\par
     	\centro
     	\par
     	\departamento
     	\par
     	\curso
     	\par
     	\vspace{08pt}
     	\text{Prova 3 - Algebra Linear}\\
     	\text{Prof. Fernando R. L. Contreras}	
     \end{center}
     
     \begin{flushleft}
     	\textbf{Aluno(a)}:
     \end{flushleft}
 
 \begin{itemize}
 	\item[1.]  Determinar a equação reduzida e esboce o gráfico da cônica representada pela equação:\\ $11x^{2}-24xy+4y^{2}+20x-40y-20=0$.
 \end{itemize}
 \begin{itemize}
 	\item[2.] Identificar a seguinte quádrica $7x^{2}+6y^{2}+5z^{2}-4xy-4yz-18=0$ utilizando os conceitos álgebra linear.
\end{itemize}
 \begin{itemize}
 	\item [3.] Dê a solução geral do seguinte sistema de equações diferenciais
 	$\begin{cases}
 	\frac{dx}{dt}=3x-4y,\\
 	\frac{dy}{dt}=x-y
 	\end{cases}$.
 \end{itemize}
 \begin{itemize}
 	\item[4.]. Seja $V$ o espaço vetorial de dimensão finita e sejam as funções $F:V\longrightarrow \mathbb{R}$ e $G: V \times V \longrightarrow \mathbb{R}$ tal que: $G(u,v)=F(u+v)-F(u)-F(v)$. Supondo que $G$ é forma bilinear e que $F(\alpha u)=\alpha^{2}F(u)$, para todo $\alpha\in \mathbb{R}$ e $u\in V$, demostrar que $F$ é uma forma quadrática e determinar a forma bilinear da qual provém.
\end{itemize}
\begin{itemize}
	\item[Opcional]. Uma forma quadrática $T$ é chamada positiva definida, se para todo $v\neq 0$, $T(v)>0$. Como devem ser os autovalores da matriz de uma forma quadrática positiva definida? \textbf{Justifique.}.
	
\end{itemize}	
\flushbottom
\flushright

\end{document}

