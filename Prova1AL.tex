\documentclass[oneside,a4paper,11pt]{article}
\usepackage[english,brazilian]{babel}
\usepackage[alf]{abntex2cite}
\usepackage[utf8]{inputenc}
\usepackage[T1]{fontenc}

\usepackage{lastpage}			  % Usado pela Ficha catalográfica
\usepackage{indentfirst}		  % Indenta o primeiro parágrafo de cada
\usepackage[top=20mm, bottom=20mm, left=20mm, right=20mm]{geometry}
\usepackage{framed}
\usepackage{booktabs}

\usepackage{float}
\usepackage{color}				  % Controle das cores
\usepackage{graphicx}			  % Inclusão de gráficos
\usepackage{microtype} 		      % para melhorias de justificação
\usepackage{booktabs}
\usepackage{multirow}
\usepackage[table]{xcolor}
\usepackage{subfig}
\usepackage{epstopdf}
\usepackage{hyperref}

\usepackage[mathcal]{eucal}
\usepackage{amsmath}               % great math stuff
\usepackage{amsfonts}              % for blackboard bold, etc
\usepackage{amsthm}                % better theorem environments
\usepackage{amssymb}
\usepackage{mathrsfs}
\DeclareMathAlphabet{\mathpzc}{OT1}{pzc}{m}{it}
\usepackage{undertilde}            % botar tilde embaixo da letra
\usepackage{mathptmx}          % fonte
\usepackage{graphicx}
\graphicspath{{./Figuras/}}    
\definecolor{shadecolor}{rgb}{0.8,0.8,0.8}


%FAZ EDICOES AQUI (somente no conteudo que esta entre entre as ultimas  chaves de cada linha!!!)
\newcommand{\universidade}{Universidade Federal de Pernambuco}
\newcommand{\centro}{Centro Acadêmico do Agreste}
\newcommand{\departamento}{Núcleo de Tecnologia}
\newcommand{\curso}{Engenharia Civil}
\newcommand{\professor}{Fernando Contreras}
\newcommand{\disciplina}{Algebra Linear}
%ATE AQUI !!!

\begin{document}
	\pagestyle{empty}
	
	\begin{center}
	\includegraphics[width=\linewidth/6]{logoUFPE.jpg}%LOGOTIPO DA INSTITUICAO
	 	\vspace{0pt}
	 	
		\universidade
		\par
		\centro
		\par
		\departamento
		\par
		\curso
		\par
		\vspace{01pt}
		\large \textbf{Primeira Prova}
		
	\end{center}
	
	\vspace{0pt}
	
	\begin{tabular}{ |l|p{12cm}| }
		
		\hline
		\multicolumn{2}{|c|}{\textbf{Dados de Identificação}} \\
			\hline
		Disciplina:        &    \disciplina          \\
		\hline
		Professor:         &    \professor           \\
	\hline
	Aluno(a):         &\\
	
		\hline
	\end{tabular}
	
	\vspace{10pt}
\begin{itemize}
	
	\item[1.] Seja $V=\left\lbrace (x,y,z)\in \mathbb{R}^{3}: z>0 \right\rbrace $ o espaço vetorial com as operações:
	$(x,y,z)\oplus(a,b,c)=(x+a+2,y+b,zc)$, $(x,y,z), (a,b,c)\in V$;
	$\alpha\odot(x,y,z)=(\alpha x+2(\alpha-1), \alpha y, z^{\alpha}), (x,y,z)\in V, \alpha \in \mathbb{R}$.
	\begin{itemize}
	\item[(a)]\textbf{(0.5)} Calcule $3\odot(-1,0,2)\oplus 2\odot(2,-1,3)$
	\item[(b)]\textbf{(0.5)} Determine o vetor nulo desse espaço vetorial.
	\item[(c)]\textbf{(0.5)} Determine o simétrico ou posto de $(x,y,z)$.
	\item[(d)]\textbf{(0.5)} Verifique se $W=\left\lbrace (x,y,z): x=-2\right\rbrace$ é um subespaço vetorial de $V$.
	\end{itemize}
\end{itemize}

\begin{itemize}
	\item[2.] Seja $V=M_{2\times2}(\mathbb{R})$ com as operações usuais e seja
	$W=\left\lbrace B\in V: B \quad \text{comuta com}\quad A=\left[\begin{array}{rr}
	1&2\\
	0&3
	\end{array}\right] \right\rbrace $.
	\begin{itemize} 
	\item[(a)]\textbf{(1.0)} Mostre que se $B\in W$, então existem constantes reais $\alpha$ e $\beta$ tais que $B=\alpha A+\beta I$, onde $I$ é a matriz identidade.
	\item[(b)]\textbf{(1.0)} Por que $W$ é um subespaço vetorial de $V$? Exiba uma base para $W$. Justifique adequadamente.
    \end{itemize}
\end{itemize}	
\begin{itemize}	
	\item[3.] Uma transformação $T:E\longrightarrow F$, entre espaços vetoriais, chama-se \textit{afim} quando se tem $T((1-t)u+tv)=(1-t)T(u)+tT(u)$ para quaisquer $u,v \in E$ e $t\in \mathbb{R}$. Dada a transformação afim $T:E\longrightarrow F$, prove que:\\
	\begin{itemize}
	\item[(a)]\textbf{(1.0)} Supondo ainda que $T(0)=0$, a relação $T(\frac{1}{2}(u+v))=\frac{1}{2}(T(u)+T(v))$, implica que $T(u+v)=T(u)+T(v)$ para quaisquer $u,v\in E$. 
	\item[(b)]\textbf{(1.0)} Para todo $b\in F$, a transformação $S:E\longrightarrow F$, definida por $S(v)=T(v)+b$ também é \textit{afim}.
    \end{itemize}
\end{itemize}
\begin{itemize} 
	\item[4.]\textbf{(2.0)} O produto vetorial de dois vetores $v=(x_{1},y_{1},z_{1})$ e $ w=(x_{2}, y_{2}, z_{2})$ em $\mathbb{R}^{3}$ é, por definição, o vetor $v\times w= (y_{1}z_{2}-z_{1}y_{2}, z_{1}x_{2}-x_{1}z_{2},x_{1}y_{2}-y_{1}x_{2})$. Fixado o vetor $u=(a,b,c)$, determine a matriz, relativamente à base canônica, do operador $A:\mathbb{R}^{3} \longrightarrow \mathbb{R}^{3}$, definido por $A(v)=v\times u$. Descreva geometricamente o núcleo desse operador e obtenha a equação da sua imagem.
\end{itemize}
\begin{itemize}
	\item[5.]\textbf{(2.0)} Considere $\mathbb{R}^{3}$ com as operações usuais e sejam $\alpha=\left\lbrace (1,1,1), (1,1,0),(1,0,0) \right\rbrace$ e \\ $A=\left[\begin{array}{rrr}
	1&0&1\\
	-1&1&0\\
	0&1&-1
	\end{array}\right]$.\\
	Determine a base $\beta=\left\lbrace v^{1}, v^{2}, v^{3}\right\rbrace $ de $\mathbb{R}^{3}$ tal que $A=[I]_{\beta}^{\alpha}$.
\end{itemize}

	\flushbottom
	\flushright
     Êxitos...!!!
\end{document}
