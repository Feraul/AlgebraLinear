%% This document created by Scientific Word (R) Version 3.0

%\documentclass[thmsa,11pt]{article}
\documentclass[12pt,openright,oneside,a4paper,english,french,spanish,brazil]{abntex2}

% ---
% Pacotes básicos 
% ---
\usepackage{lmodern}			  % Usa a fonte Latin Modern			
\usepackage[T1]{fontenc}		  % Selecao de codigos de fonte.
\usepackage[utf8]{inputenc}	      % Codificacao do documento (conversão automática dos acentos)

\usepackage{lastpage}			  % Usado pela Ficha catalográfica
\usepackage{indentfirst}		  % Indenta o primeiro parágrafo de cada seção.
\usepackage{color}				  % Controle das cores
\usepackage{graphicx}			  % Inclusão de gráficos
\usepackage{microtype} 		      % para melhorias de justificação
\usepackage{booktabs}
\usepackage{multirow}
\usepackage[table]{xcolor}
\usepackage{subfig}
\usepackage{epstopdf}
\usepackage{hyperref}
\usepackage[alf,bibjustif,abnt-etal-cite=0,abnt-etal-list=0]{abntex2cite} % Citações padrão ABNT
\usepackage[mathcal]{eucal}
\usepackage{amsmath}               % great math stuff
\usepackage{amsfonts}              % for blackboard bold, etc
\usepackage{amsthm}                % better theorem environments
\usepackage{amssymb}
\usepackage{mathrsfs}
\DeclareMathAlphabet{\mathpzc}{OT1}{pzc}{m}{it}
\usepackage{undertilde}            % botar tilde embaixo da letra
\usepackage{mathptmx}          % fonte
\usepackage{graphicx}
%tCIDATA{OutputFilter=latex2.dll}
%tCIDATA{TCIstyle=article/art4.lat,lart,article}
%tCIDATA{CSTFile=article.cst}
%tCIDATA{Created=Tue Jan 16 17:54:47 2001}
%tCIDATA{LastRevised=Sun Feb 16 19:37:46 2003}
%tCIDATA{<META NAME="GraphicsSave" CONTENT="32">}
\setlength{\textheight}{19.5cm}
\newtheorem{teorema}{Teorema}[subsection]
\newtheorem{obs}[teorema]{Observa\c{c}\~{a}o}
\newtheorem{defini}[teorema]{Defini\c{c}\~{a}o}
\newtheorem{prop}[teorema]{Proposi\c{c}\~{a}o}
\newtheorem{corolario}[teorema]{Corol\'{a}rio}
\newtheorem{lema}[teorema]{Lema}
\newtheorem{exercicio}[teorema]{Exercício}
\newenvironment{demons}{\noindent{\bf Demonstração:} }{\hfill $\Box$ \newline}
\newenvironment{demonsembox}{\noindent{\bf Demonstração:} }{}
\newenvironment{exemplos}{\noindent {\bf Exemplos:} }{\hfill $\Box$ \newline}
\newenvironment{exemplo}{\vspace{12pt} \noindent{\bf Exemplo:} }{\hfill $\Box$ \newline}
\newenvironment{exemplosembox}{\vspace{12pt} \noindent {\bf Exemplos:} }{}
\newenvironment{demdoteo}{\noindent {\bf Demonstração do Teorema}}{\hfill $\Box$ \newline}
\newenvironment{demdolem}{\noindent {\bf Demonstração do Lema}}{\hfill $\Box$ \newline}
\newcommand{\adj}{\mathop{\rm ad}\nolimits}
\newcommand{\diag}{\mathop{\rm diag}\nolimits}
\newcommand{\esseo}{\mathop{\rm SO}\nolimits}
\newcommand{\gera}{\mathop{\rm ger}\nolimits}
\newcommand{\ident}{\mathop{\rm id}\nolimits}
\newcommand{\imag}{\mathop{\rm im}\nolimits}
\newcommand{\partre}{\mathop{\rm Re}\nolimits}
\newcommand{\partim}{\mathop{\rm Im}\nolimits}
\newcommand{\trac}{\mathop{\rm tr}\nolimits}
\hyphenation{con-si-de-ran-do-se ca-sa}

\begin{document}

\author{Algebra Linear\\ Engenharia Civil, 2do Periodo}
\title{Lista 1}

\maketitle
\begin{enumerate}[]
%\item Quais são as coordenadas da função $3sen(t)+5cos(t)=f(t)$ em relação à base $\left\lbrace sen(t), cos(t) \right\rbrace$?

\item Exprima cada vetor do conjunto $\left\lbrace u, v, w, z \right\rbrace \subset E$ como combinação linear dos vetores $\left\lbrace w, u+3z, v-2u+3w,5z \right\rbrace$.

\item Obtenha uma base para o subespaço vetorial gerado por cada um dos seguintes conjuntos e, consequentemente, determine a dimensão daquele subespaço:\\

$(a)$ $\left\lbrace (1,2,3,4), (3,4,7,10), (2,1,3,5) \right\rbrace$\\

$(b)$ $\left\lbrace (1,3,5), (-1,3,-1), (1,21,1) \right\rbrace$ \\

$(c)$ $\left\lbrace (1,2,3), (1,4,9), (1,8,27) \right\rbrace$ 
\item
Prove que o sistema:
\begin{center}
	$x+2y+3z-3t=a$\\
	$2x-5y-3z+12t=b$\\
	$7x+y+8z+5t=c$\\
\end{center}
Admite solução se, e somente se, $37a+13b=9c$. Ache a solução geral do sistema quando $a=2$ e $b=4$.

\item Ache uma base para o núcleo de cada uma das transformações lineares a seguir:\\
$(a)$ $F:R^{3}\longrightarrow R^{3}$, $F(x,y,z)=(-3y+4z,3x-5z,-4x+5y)$\\

$(b)$ $C:R^{4}\longrightarrow R^{3}$, $C(x,y,z,t)=(2x+y-z+3t,x-4y+2z+t,y+4z-t)$
\item Use escalonamento para resolver o seguinte Sistema Linear
\begin{center}
	$x+y+t=0$\\
	$x+2y+z+t=1$\\
	$3x+3y+z+2t=-1$\\
	$y+3z-t=3$
\end{center}

\item Obtenha os numeros $a$, $b$, $c$ tais que $ax+by+cz=0$ seja a equação do plano gerado pelas colunas da matriz
\begin{equation*}
\left[\begin{array}{rrr}
1&1&1\\
1&2&3\\
2&3&4
\end{array}\right]
\end{equation*}

\item \textbf{Definição}. Seja $P:R^{2}\longrightarrow R^{2}$ a projeção ortogonal sobre uma certa certa reta $r$. Para todo $v$ sobre a reta $r$, tem-se $P(v)=v$. Assim, para qualquer $v\in R^{2}$ tem-se $P(P(v))=P(v)$, pois $P(v) $ esta sobre a reta $r$. Determine a matriz da projeção $P:R^{2}\longrightarrow R^{2}$, $p(x,y)=(x,0)$ relativamente à base $\left\lbrace u,v\right\rbrace \subset R^{2} $, onde $u=(1,1)$ e $v=(1,2)$.
\item Usando a \textbf{definição} anterior , prove que o operador $P:R^{2}\longrightarrow R^{2}$, dado por $P(x,y)=(-2x-4y,\frac{3}{2}x +3y)$ é a projeção sobre uma reta. Determine o núcleo e a imagem de $P$.
\item Encontre os números $a$, $b$, $c$, $d$ de modo que o operador $F: R^{2} \longrightarrow R^{2}$, dado por $F(x,y)=(ax+by,cx+dy)$ tenha como núcleo a reta $y=3x$.
\item Mostre que os vetores $u=(1,1,1)$, $v=(1,2,3)$ e $w=(1,4,9)$ formam uma base de $R^{3}$. Exprima cada um dos vetores $e_{1}$, $e_{2}$, $e_{3}$ da base canônica de $R^{3}$ como combinação linear de $u$, $v$ e $w$. 
\item Diz-se que uma função $f:X\longrightarrow R$ é limitada quando existe $k>0$ tal que $\left| f(x) \right|\leq k$ para todo $x\in X$. Prove que o conjunto das funções limitadas é um subespaço vetorial de $F(X;R)$.\\ 
Note que, $F(X;R)$ é o espaço vetorial das funções reais de variável real $f:X\longrightarrow R$.
\item Seja $E$ um espaço vetorial e $u, v \in E$. O segmento de reta de extremidades $u,v$ é, por definição, o conjunto 
$[u,v]=\left\lbrace (1-t)u+tv; 0\leq t\leq1\right\rbrace $.\\
Um conjunto $X\subset E$ chama-se convexo quando $u,v \in X \Rightarrow [u,v]\subset X$.\\
Prove que dados $a$, $b$, $c\in R$, o conjunto $X=\left\lbrace(x,y)\in R^{2}; ax+by\leq c\right\rbrace$ é conjunto convexo em $R^{2}$   
\end{enumerate}
\begin{flushright}
	Êxitos...!
\end{flushright}
\end{document}