
%%%%%%%%%%%%%%%%%%%%%%%%%%%%%%%%%%%%%%%%%%%%%%%%%%%%%%%%%%%%%%%%%%%%%%%%
%    Option test file, will be created during the first LaTeX run:
\begin{filecontents}{exercise.thm}
\def\th@exercise{%
  \normalfont % body font
  \thm@headpunct{:}%
}
\end{filecontents}
%%%%%%%%%%%%%%%%%%%%%%%%%%%%%%%%%%%%%%%%%%%%%%%%%%%%%%%%%%%%%%%%%%%%%%%%

\documentclass[12pt,openright,oneside,a4paper,english,french,spanish,brazil]{article}
% ---
% Pacotes básicos 
% ---
\usepackage{lmodern}			  % Usa a fonte Latin Modern			
\usepackage[T1]{fontenc}		  % Selecao de codigos de fonte.
\usepackage[utf8]{inputenc}	      % Codificacao do documento (conversão automática dos acentos)
\usepackage[top=20mm, bottom=20mm, left=20mm, right=20mm]{geometry}
\usepackage{lastpage}			  % Usado pela Ficha catalográfica
\usepackage{indentfirst}		  % Indenta o primeiro parágrafo de cada seção.
\usepackage{color}				  % Controle das cores
\usepackage{graphicx}			  % Inclusão de gráficos
\usepackage{microtype} 		      % para melhorias de justificação
\usepackage{booktabs}
\usepackage{multirow}
\usepackage[table]{xcolor}
\usepackage{subfig}
\usepackage{epstopdf}
\usepackage{hyperref}
\usepackage[mathcal]{eucal}
\usepackage{amsmath}               % great math stuff
\usepackage{amsfonts}              % for blackboard bold, etc
\usepackage{amsthm}                % better theorem environments
\usepackage{amssymb}
\usepackage{mathrsfs}
\DeclareMathAlphabet{\mathpzc}{OT1}{pzc}{m}{it}
\usepackage{undertilde}            % botar tilde embaixo da letra
\usepackage{mathptmx}          % fonte
\usepackage{latexsym}
\usepackage{makeidx}            % para definir o índice
\usepackage{epsfig}             % para introduzir figuras no formato eps
\usepackage{graphicx,color}     % permite a inclusao de figuras
\usepackage{verbatim}
\usepackage{gensymb}
\usepackage{titling}
\newcommand{\subtitle}[1]{%
	\posttitle{%
		\par\end{center}
	\begin{center}\Large#1\end{center}
	\vskip0.5em}%
}



\newtheorem{df}{Definição}
\newtheorem{ex}{Exemplo}
\newtheorem{teo}{Teorema}

\newtheoremstyle{note}% name
  {3pt}%      Space above
  {3pt}%      Space below
  {}%         Body font
  {}%         Indent amount (empty = no indent, \parindent = para indent)
  {\itshape}% Thm head font
  {:}%        Punctuation after thm head
  {.5em}%     Space after thm head: " " = normal interword space;
        %       \newline = linebreak
  {}%         Thm head spec (can be left empty, meaning `normal')

\theoremstyle{note}
\newtheorem{note}{Note}

\newtheoremstyle{citing}% name
  {3pt}%      Space above, empty = `usual value'
  {3pt}%      Space below
  {\itshape}% Body font
  {}%         Indent amount (empty = no indent, \parindent = para indent)
  {\bfseries}% Thm head font
  {.}%        Punctuation after thm head
  {.5em}%     Space after thm head: " " = normal interword space;
        %       \newline = linebreak
  {\thmnote{#3}}% Thm head spec

\theoremstyle{citing}
\newtheorem*{varthm}{}% all text supplied in the note

\newtheoremstyle{break}% name
  {9pt}%      Space above, empty = `usual value'
  {9pt}%      Space below
  {\itshape}% Body font
  {}%         Indent amount (empty = no indent, \parindent = para indent)
  {\bfseries}% Thm head font
  {.}%        Punctuation after thm head
  {\newline}% Space after thm head: \newline = linebreak
  {}%         Thm head spec

\theoremstyle{break}
\newtheorem{bthm}{B-Theorem}

\theoremstyle{exercise}
\newtheorem{exer}{Exercise}

\swapnumbers
\theoremstyle{plain}
\newtheorem{thmsw}{Theorem}[section]
%\newtheorem{corsw}[thm]{Corollary}
\newtheorem{propsw}{Proposition}
%\newtheorem{lemsw}[thm]{Lemma}

%    Because the amsmath pkg is not used, we need to define a couple of
%    commands in more primitive terms.
\let\lvert=|\let\rvert=|
\newcommand{\Ric}{\mathop{\mathrm{Ric}}\nolimits}

%    Dispel annoying problem of slightly overlong lines:
\addtolength{\textwidth}{8pt}

\title{ \textbf{Notas de Aula}}
\subtitle{\textbf{Álgebra Linear}}
\author{\textbf{Fernando Contreras}\\
	\large Nucleo de Tecnologia\\
	Universidade Federal de Pernambuco (UFPE)}



\begin{document}
	\begin{center}
		Universidade Federal de Pernambuco (UFPE)\\
		Centro Acadêmico do Agreste\\
		Núcleo de Tecnologia\\
		
		Lista 3 de Álgebra Linear\\
		Prof. Fernando RL Contreras
	\end{center}


Sejam os seguintes problemas

\begin{itemize}
	\item[1.] Sendo $u=(x_{1},x_{2})$ e $v=(y_{1},y_{2})$ são vetores genéricos do $\mathbb{R}^{2}$, definamos $<u,v>=\frac{x_{1}y_{1}}{a^{2}}+\frac{x_{2}y_{2}}{b^{2}}$ com $a,b\in \mathbb{R}$ fixos e não nulos. Provar que $<u,v>$ define um produto interno sobre $\mathbb{R}^{2}$.
	%Exer 6, pag. 161, Caliolli 
\end{itemize}
\begin{itemize}
	\item[2.] Em $P_{2}(\mathbb{R})$ com produto interno dado por $<f(t),g(t)>=\int_{0}^{1}dt$ calcule a norma de $f(t)$ nos seguintes casos:
	\begin{itemize}
		\item $f(t)=t$.
		\item $f(t)=-t^{2}+1$.
	\end{itemize}
\end{itemize}
% Exer 8, pag. 162 Caliolli
\begin{itemize}
	\item [3.] Em $P_{2}(\mathbb{R})$ com o produto interno definido por: $<f(t),g(t)>=\int_{0}^{1}f(t)g(t)dt$ 
	\begin{itemize}
		\item Ortonormalize a base $\{1,1+t,2t^{2}\}$.
		\item Ache o complemento ortogonal do sub-espaço $W=[5,1+t]$
	\end{itemize}
% Exer 7, pag. 185 Caliolli
\end{itemize}
\begin{itemize}
	\item[4.] Sejam $U$ e $V$ sub-espaços vetoriais de um espaço euclidiano de dimensão finita. Prove que  $(U\cap V)^{\bot}=U^{\bot}+V^{\bot}$.
% Exer 15, pag. 185 Caliolli
\end{itemize}
\begin{itemize}
	\item[5.] Consideremos o $\mathbb{R}^{4}$ munido do produto interno usual e seja $W=\{(x,y,z,t)\in \mathbb{R}^{4}|\quad x+y=0 \quad\text{e}\quad 2x-y+z=0\}$. Determine uma base ortonormal de $W$ e uma base ortonormal de $W^{\bot}$.
% Exer 8, pag. 185 Caliolli	
\end{itemize}

\begin{itemize}
	\item[6.] Considere a seguinte transformação linear do $\mathbb{R}^{3}$ no $\mathbb{R}^{2}$: $F(x,y,z)=(x-y-z,2z-x)$. Determine uma base ortonormal de $Ker(F)$ em relação ao produto interno usual.
% Exer 5, pag. 184 Caliolli		
\end{itemize}
\begin{itemize}
	\item[7.] Achar a projeção ortogonal de $(1,1,1,1)\in\mathbb{R}^{4}$ sobre o subespaço $W=[(1,1,0,0),(0,0,1,1)]$, usando o produto interno usual do $\mathbb{R}^{4}$.
% Exer 18, pag. 181 Caliolli		
\end{itemize}
\begin{itemize}
	\item[8.]  Seja $V$ um espaço euclidiano. Demostrar que a função $N:V\rightarrow \mathbb{R}$ definida por $N(u)=<u,u>$. Verifica que:
	\begin{itemize}
		\item $N(\alpha u)=\alpha^{2}N(u)$.
		\item $N(u+v)-N(u)-N(v)=2<u,v>$.
		\item $\frac{1}{4}N(u+v)-\frac{1}{4}N(u-v)=<u,v>$.
	\end{itemize}
%Exer 7-35, pag. 258, Armando Rojo 
\end{itemize}
\begin{itemize}
	\item[9.] Considere o subespaço $W$ de $\mathbb{R}^{3}$ gerado por $v_{1}=(1,0,0)$, $v_{2}=(0,1,1)$ e $v_{3}=(1,-1,-1)$. Sendo $<,>$ o produto interno usual (a) Ache $W^{\bot}$; (b) Exiba uma transformação linear $T:\mathbb{R}^{3}\rightarrow\mathbb{R}^{3}$ tal que $Im(T)=W$ e $Ker(T)=W^{\bot}$.
	% Exer 10, pag. 248 Boldrini 
\end{itemize}
\begin{itemize}
	\item[10.]Podemos definir uma "distancia" entre dois ponto $P=(x_{1},y_{1})$ e $Q=(x_{2},y_{2})$ do plano por $d(P,Q)=|x_{2}-x_{1}|+|y_{2}-y_{1}|$. Verifique se a aplicação dada por $<(x_{1},y_{1}),(x_{2},y_{2})>=d(P,Q)$ define um produto interno no plano.
	%Exer 17 pag. 249  Boldrini
\end{itemize}
\begin{itemize}
	\item[11.] Seja $E$ um espaço vetorial com produto interno. Prove que para quaisquer $u,v\in E$, tem-se $| \left\| u \right\| - \left\| v \right\| | \leq \left\| u-v \right\| $.
	% Exer 10.31, pag. 136 Elon Lages 
\end{itemize}
\begin{itemize}
	\item [12.] Seja $<,>$ um produto interno no espaço vetorial $F$. Dado um isomorfismo $A: E\rightarrow F$, ponha $[u,v]=<A(u),A(v)>$ para quaisquer $u,v \in E$. Prove que $[,]$ é um produto interno em $E$.
	% Exer 10.19, pag. 134 Elon Lages 
\end{itemize}




\begin{itemize}
	\item [13.] Seja $A=\begin{bmatrix}
	3    & -2  \\
	-1     & 2  
	\end{bmatrix}$ e $P(X)=X^{2}-1$. Diagonalizar $P(A)$, se possível.
\end{itemize}
\begin{itemize}
	\item [14.] Seja $T:V \longrightarrow V $ a) Se $\lambda=0$ é autovalor de $T$, mostre que $T$ não é injetora. b) A reciproca é verdadeira? Ou seja, se $T$ não é injetora, $\lambda=0$ é autovalor de $T$.
	% Exerc 25 pag 196 Boldrini
\end{itemize}
\begin{itemize}
	\item [15.] Seja $A=\begin{bmatrix}
	0    & 2  \\
	1    & 1  
	\end{bmatrix}$ a) Ache os autovalores de $A$ e $A^{-1}$. b) Quais são os autovetores correspondentes?
	% Exerc 22 pag 196 Boldrini
\end{itemize}
\begin{itemize}
	\item [16.] Suponha que $v\in V$ seja autovetor de $T:V \longrightarrow V  $e $S:V \longrightarrow V $, ao mesmo tempo com autovalores $\lambda_{1}$ e $\lambda_{2}$ respectivamente. Ache os autovalores e autovetores de a) $S+T$ e b) $S\circ T$.
		% Exerc 24 pag 196 Boldrini
\end{itemize}
\begin{itemize}
	\item [17.] Ache os autovalores e autovetores correspondentes das transformações lineares dadas:
	 \begin{itemize}
	 	\item $T:P_{2}\rightarrow P_{2}$ tal que $ T(ax^{2}+bx+c)=ax^{2}+cx+b$.
	 	\item $T:\mathbb{R}^{2}\rightarrow \mathbb{R}^{2}$ tal que $ T(x,y,z,w)=(x,x+y,x+y+z,x+y+z+w)$.
	 \end{itemize}
 	% Exerc 5 e 7 pag 195 Boldrini
\end{itemize}
\begin{itemize}
	\item [18.] Diz-se que um operador linear $T: V \rightarrow V$ é idempotente se $T^{2}=T$ (isto é, se $T\circ T(v)=T(v)$ para todo $v\in V$). a) Seja $T$ idenpotente. Ache seus autovalores. b) Mostre que um operador linear idenpotente é diagonalizável. 
	% Exerc 5 e 7 pag 195 Boldrini
\end{itemize}
\begin{itemize}
	\item [19.] Seja $A$ matriz $3X3$ triangular superior, com todos os seus elementos acima da diagonal distintos e não nulos.
	$A=\begin{bmatrix}
	a    & b & c \\
	0    & d &e \\
	0    & 0 & f
	\end{bmatrix}$ Quais são os autovalores e autovetores de $A$?.
	% Exerc 4 pag 213 Boldrini
\end{itemize}
\begin{itemize}
	\item [20.] Mostre que $A=\begin{bmatrix}
	1    & 2 \\
	3    & 2
	\end{bmatrix}$ é semelhante à matriz $B=\begin{bmatrix}
	4    & 0 \\
	0    & -1
	\end{bmatrix}$.
	% Exerc 8 pag 214 Boldrini
\end{itemize}
\begin{itemize}
	\item [21.] Utilize a forma diagonal para encontrar $A^{n}$ no seguinte caso  $A=\begin{bmatrix}
	-3    & 4 \\
	-1    & 2
	\end{bmatrix}$.
	% Exerc 14 pag 215 Boldrini
\end{itemize}
\end{document}
